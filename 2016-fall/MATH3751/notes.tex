\documentclass{article}
\usepackage{amsmath}
\usepackage{amsthm}
\usepackage{amssymb}

\newtheoremstyle{break}% name
  {}%         Space above, empty = `usual value'
  {}%         Space below
  {\itshape}% Body font
  {}%         Indent amount (empty = no indent, \parindent = para indent)
  {\bfseries}% Thm head font
  {.}%        Punctuation after thm head
  {\newline}% Space after thm head: \newline = linebreak
  {}%         Thm head spec

\theoremstyle{break}
\newtheorem{theorem}{Theorem}

\title{MATH3751 Lecture Notes}
\author{Ricky Elrod}
\begin{document}
\maketitle

\section{24 August 2016}

We start with $\mathbb{Z}^+$ or $\mathbb{N}$\textbackslash$\{0\}$, the set of
natural numbers (1, 2, 3, \ldots). We have operations e.g. $m + n$, $m \cdot
n$, along with an order relation, $m < n$.

However, when we try to discuss subtraction and division, we discover that they
aren't closed operations on this set. For example, $3 - 5 = -2 \not\in
\mathbb{Z}^+$, and $\frac{1}{2} \not\in \mathbb{Z}^+$.

So we define the integers ($\mathbb{Z}$) which include negative numbers and $0$,
e.g. ($\ldots$, -3, -2, -1, 0, 1, 2, 3, $\ldots$), and we find that now
subtraction is always defined, but we still run into problems with division.

So we define $\mathbb{Q}$, the set of all rational numbers; that is $\frac{m}{n}
\ni n \not= 0$ where $m, n \in \mathbb{Z}$. Now division is also always defined
and we have facts such as:
\begin{itemize}
\item $\frac{m}{n} = \frac{a}{b} \iff m \cdot b = n \cdot a$
\item $\frac{m}{n} + \frac{a}{b} = \frac{mb}{nb} + \frac{na}{nb} = \frac{mb +
    na}{nb}$
\item $\frac{m}{n} \cdot \frac{a}{b} = \frac{ma}{nb}$
\item $\frac{m}{n} < \frac{a}{b} \iff$ if $\forall n, b > 0,\ m \cdot b < n \cdot a$
\end{itemize}

However, there are numbers that are not rational.

\begin{theorem}[$\sqrt{2}$ is irrational]
  We proceed by contradiction. Assume that $\sqrt{2}$ is rational. Then $\exists
  m, n \in \mathbb{Z} \ni \sqrt{2} = \frac{m}{n}$, where $n \not= 0$. Assume
  gcd($m$, $n$) = $1$. Then $\sqrt{2} = \frac{m}{n} \implies \sqrt{2} \cdot n =
  m \implies (\sqrt{2} \cdot n)^2 = m^2 \implies 2n^2 = m^2$. So $m^2$ is even,
  thus $m$ is even. So $m = 2l$ for some $l \in \mathbb{Z}$. Now $\sqrt{2} =
  \frac{2l}{n} \implies \sqrt{2n} = 2l \implies 2n^2 = 4l^2 \implies n^2 = 2l^2
  \implies n^2$ is even $\implies n$ is even. Thus we have a contradiction since
  we assumed gcd($n$, $m$) = $1$. \qed
\end{theorem}

\subsection{Properties of $\mathbb{R}$}
\subsubsection{Additive}
\begin{itemize}
\item $\forall a, b \in \mathbb{R}, a + b \in \mathbb{R}$ and $a + b = b + a$
\item $\forall a, b, c \in \mathbb{R}, (a+b)+c = a+(b+c)$
\item $\exists! 0 \in \mathbb{R} \ni \forall a \in \mathbb{R} a + 0 = 0 + a = a$
\item $\forall a \in \mathbb{R}, \exists -a \in \mathbb{R} \ni a + (-a) = 0$
\end{itemize}

\subsubsection{Multiplicative}
\begin{itemize}
\item $\forall a, b \in \mathbb{R}, a \cdot b \in \mathbb{R}$ and $a \cdot b = b \cdot a$
\item $\forall a, b, c \in \mathbb{R}, (a \cdot b) \cdot c = a \cdot (b \cdot c)$
\item $\exists! 1 \in \mathbb{R} \ni \forall a \in \mathbb{R} a \cdot 1 = 1 \cdot a = a$
\item $\forall a \not= 0 \in \mathbb{R}, \exists a^{-1} \in \mathbb{R} \ni a \cdot a^{-1} = 1$
\end{itemize}

\subsubsection{Additive / Multiplicative}
\begin{itemize}
\item $\forall a, b, c \in \mathbb{R}, (a + b)c = a \cdot c + b \cdot c$
\end{itemize}

\subsubsection{Order Structure ($x < y$, $x \le y$)}
\begin{itemize}
\item $\forall a, b \in \mathbb{R}$, we have exactly one of the following:
  \begin{itemize}
  \item $a = b$
  \item $a < b$
  \item $a > b$
  \end{itemize}
\item $\forall a, b, c \in \mathbb{R}, a < b \land b < c \implies a < c$
\item $\forall a, b, c \in \mathbb{R}, a < b \implies a + c < b + c$
\item $\forall a, b \in \mathbb{R}, a < b \implies a \cdot c < b \cdot c$, if $c
  > 0 \in \mathbb{R}$
\end{itemize}

\subsection{Bounds}
If $E$ is a subset of $\mathbb{R}$:

\begin{itemize}
\item $M$ is said to be an \textbf{upper bound} of $E$ if $\forall x \in E, x
  \le M$
\item $m$ is said to be a \textbf{lower bound} of $E$ if $\forall x \in E, m \le
  x$
\item If $\forall x \in E, \exists M \in E \ni x \le M$, $M$ is said to be a
  \textbf{max} and written $M = $max($E$).
\item If $\forall x \in E, \exists m \in E \ni m \le x$, $M$ is said to be a
  \textbf{min} and written $m = $min($E$).
\end{itemize}

\subsubsection{Definitions}

\begin{itemize}
\item A set with an upper bound and a lower bound is said to be
  \textbf{bounded}.

\item \textbf{Definition (Least Upper Bound/Supremum)}: Given $E$ a subset of
$\mathbb{R}$ that is bounded above and nonempty, if $M$ is the least of all
upper bounds, $M$ is the ``least upper bound'' or ``supremum'' and written
lub($E$) or sup($E$).

\item \textbf{Definition (Greatest Lower Bound/Infimum)}: Given $E$ a subset of
$\mathbb{R}$ that is bounded below and nonempty, if $m$ is the greatest of all
lower bounds, $m$ is the ``greatest lower bound'' or ``infimum'' and written
glb($E$) or inf($E$).

\item \textbf{Definition (Completeness Axiom)}: Every nonempty set of real
  numbers that is bounded above has a least upper bound.
\end{itemize}

\end{document}
